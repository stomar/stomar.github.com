\documentclass[a4paper,12pt,halfparskip]{scrartcl}

\usepackage{ngerman}
\usepackage[utf8]{inputenc}
\usepackage[T1]{fontenc}

% Times / Helvetica
%\usepackage{mathptmx}
%\usepackage[scaled=.90]{helvet}

% Palatino / Helvetica
\usepackage{mathpazo}
\usepackage[scaled=.95]{helvet}

%%%%%%%%%%%%%%%%%%%%%%%%%%%%%%%%%%%%%%%%%%%%%%%%%%%%%%%%%%

%%%% Zur PDF-Erzeugung (Hyperlinks) benötigte Optionen
%%%% Details siehe Dokumentation von hyperref

%%%% Bitte ganz am Schluß unterbringen,
%%%% da viele Befehle neu definiert werden

%\usepackage{ae}       % zur Suche nach Worten mit Umlaut noetig
\usepackage[ps2pdf,
            bookmarks=false,
            bookmarksopen=false,%    Bookmarks anzeigen...
            bookmarksnumbered=false,% ...und numerieren
            colorlinks,%             farbige Links
            %plainpages=false,
            pagebackref=false,
            linkcolor=red,
            urlcolor=blue,
            breaklinks=false,
            pdfpagemode={None},
            pdfauthor={Marcus Stollsteimer},
            pdftitle={Disposition der Orgel in der Stiftskirche Herrenberg},
            pdfsubject={},
            pdfkeywords={},
            pdfcreator={LaTeX with hyperref package},
            pdfproducer={dvips + ps2pdf},%
           ]{hyperref}
%\usepackage[dvips]{thumbpdf}


\setlength{\parindent}{0pt}

\pagestyle{empty}

\newcommand{\p}{\phantom{1}}
\newcommand{\Fuss}{$^\prime$}

\newcommand{\bruch}[2]{$\raisebox{2pt}{\footnotesize#1}\hspace{-2pt}/\hspace{-1.5pt}\raisebox{-2pt}{\footnotesize#2}$}
\newcommand{\eh}{\bruch{1}{2}$\,$}
\newcommand{\ed}{\bruch{1}{3}$\,$}
\newcommand{\zd}{\bruch{2}{3}$\,$}
\newcommand{\df}{\bruch{3}{5}$\,$}


\begin{document}

{\LARGE\bfseries Die Orgel der Stiftskirche Herrenberg}

\vfill

\begin{minipage}[t]{.55\textwidth}
{\bfseries I. Manual (C--g$^{\mathbf3}$)}
\begin{tabbing}
\hspace{5cm}\= \kill
\mbox{}\p1. Bourdon \>16\Fuss\' \\
\p2. Principal$^*$ \>8\Fuss\' \\
\p3. Holzflöte \>8\Fuss\' \\
\p4. Octave \>4\Fuss\' \\
\p5. Rohrflöte \>4\Fuss\' \\
\p6. Quinte \>2 \zd\Fuss\' \\
\p7. Octave \>2\Fuss\' \\
\p8. Mixtur V \>2\Fuss\' \\
\p9. Zimbel III \>\eh\Fuss\' \\
10. Kornett V \>8\Fuss\'\ ab f$^{\,\circ}$ \\
11. Trompete \>8\Fuss\'
\end{tabbing}

{\bfseries II. Manual (C--g$^{\mathbf3}$)}
\begin{tabbing}
\hspace{5cm}\= \kill
12. Salicional \>8\Fuss\' \\
13. Schwebung \>8\Fuss\'\ ab c$^\circ$ \\
14. Gedeckt \>8\Fuss\' \\
15. Geigenprincipal$^*$ \>8\Fuss\' \\
16. Flauto dolce \>4\Fuss\' \\
17. Fugara \>4\Fuss\' \\
18. Waldflöte \>2\Fuss\' \\
19. Scharf IV--VI $\;$ \>1 \ed\Fuss\' \\
20. Klarinette \>8\Fuss\'
\end{tabbing}
\end{minipage}
\begin{minipage}[t]{.45\textwidth}
{\bfseries Pedal (C--f$^{\mathbf1}$)}
\begin{tabbing}
\hspace{5cm}\= \kill
21. Principalbaß \>16\Fuss\' \\
22. Subbaß \>16\Fuss\' \\
23. Octavbaß \>8\Fuss\' \\
24. Spitzflöte \>8\Fuss\' \\
25. Choralbaß \>4\Fuss\' \\
26. Posaune \>16\Fuss\' \\
27. Trompete \>8\Fuss\' \\
28. Clarine \>4\Fuss\'
\end{tabbing}

\vspace{\baselineskip}

{\bfseries III. Manual (C--g$^{\mathbf3}$)} \\
\phantom{\bfseries III.} {\bfseries neues Rückpositiv} \\
\phantom{\bfseries III.} (schwellbar)
\begin{tabbing}
\hspace{5cm}\= \kill
29. Rohrgedeckt \>8\Fuss\' \\
30. Principal$^{**}$ \>4\Fuss\' \\
31. Kleingedeckt \>4\Fuss\' \\
32. Sesquialter II \\
33. Octave \>2\Fuss\' \\
34. Quinte \>1 \ed\Fuss\' \\
35. Scharfzimbel IV \>1\Fuss\' \\
36. Basson-Hautbois \>8\Fuss\' \\
--- Tremulant ---
\end{tabbing}
\end{minipage}

\medskip

{\footnotesize
\hspace{57mm}
$^*$ teilweise Prospekt
\hspace{5mm}
$^{**}$ Prospekt, vor Schweller
}

\vfill

Zimbelstern

\smallskip

Koppeln   II/I, III/I, III/II, I/P, II/P, III/P

\smallskip

Schleif"|laden, mechanische Spieltraktur, elektrische Registertraktur.

\smallskip

Elektronische Setzerkombination mit 128 frei einstellbaren Setzern, \\
Schrittschalter vorwärts und rückwärts, Absteller.

\vfill

\begin{tabbing}
Erbauer: $\:$\=Orgelbau Rensch, Lauffen/Neckar, 1985, \\
 \>unter Verwendung des Prospekts und des Pfeifenmaterials \\
 \>der Walcker-Orgel von 1890.
\end{tabbing}

\end{document}
